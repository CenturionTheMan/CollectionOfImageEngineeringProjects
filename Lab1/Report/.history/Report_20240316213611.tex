\documentclass{article}
\input{TexBase/DocumentBase.tex}


\section{Zadanie pierwsze}
    \subsection{Cel ćwiczenia}
    Zadanie pierwsze polegało na zastosowaniu filtru górnoprzepustowego (tzw. detektora krawędzi) do obrazu. 
    W tym celu mieliśmy wykorzystać maskę:
    \[
        \begin{bmatrix}
            -1 & -1 & -1 \\
            -1 & 8 & -1 \\
            -1 & -1 & -1
        \end{bmatrix}
    \]

    \subsection{Teoria}
    Filtry obrazów są kluczowym narzędziem w dziedzinie przetwarzania obrazów, 
    służącym do modyfikacji wyglądu i charakterystyk obrazu poprzez różnorodne operacje matematyczne 
    na pikselach. Wśród podstawowych rodzajów filtrów wyróżnia się filtry górnoprzepustowe i dolnoprzepustowe.

    Filtry górnoprzepustowe, takie jak implementowany w zadaniu filtr Laplace'a, mają za zadanie podkreślać detale i krawędzie
    poprzez eliminację niskoczęstotliwościowych składowych obrazu i przepuszczanie wysokoczęstotliwościowych. 
    Działają na zasadzie mnożenia wartości pikseli przez odpowiednie współczynniki z maski, 
    co prowadzi do wyostrzenia obrazu i podkreślenia struktur. 
    
    Przykładowe filtry górnoprzepustowe to
    \begin{itemize}
        \item Filtr Laplace'a
        \item Filtr usuń średnią (ang. mean removal)
        \item Filtr HP1, HP2, HP3
    \end{itemize}

    
    Z kolei filtry dolnoprzepustowe, np. filtr Gaussa, przepuszczają składowe niskoczęstotliwościowe, 
    eliminując wysokoczęstotliwościowe. Ich działanie polega na wygładzaniu obrazu poprzez średnią lub 
    ważoną wartość pikseli w otoczeniu. W efekcie uzyskuje się efekt rozmycia, który może być wykorzystywany 
    m.in. do redukcji szumów.

    Przykładowe filtry dolnoprzepustowe to
    \begin{itemize}
        \item Filtr uśredniający
        \item Filtr kołowy
        \item Filtr piramidalny
    \end{itemize}
        
    Źródło: \cite{Lubinski2007}

    \subsection{Prezentacja wykonanego zadania}
    \begin{figure}[H]
        \centering
        \resizebox{\columnwidth}{!}{%
        \includegraphics{Img/zad1.png}%
        }
        \caption{Zdjęcie oryginalne i po nałożeniu filtru górnoprzepustowego podanego w zadaniu}
    \end{figure}

    \subsection{Wizualizację wybranych filtrów}


    \subsection{Eksperymenty z modyfikacją maski}

    \begin{figure}[H]
        \centering
        \resizebox{\columnwidth}{!}{%
        \includegraphics{Img/zad1_1.png}%
        }
        \caption{Otrzymane zdjęcie po nałożeniu maski: $\begin{bmatrix}
                                                            1 & 1 & 1 \\
                                                            1 & 8 & 1 \\
                                                            1 & 1 & 1
                                                        \end{bmatrix}$}
    \end{figure}

    \begin{figure}[H]
        \centering
        \resizebox{\columnwidth}{!}{%
        \includegraphics{Img/zad1_2.png}%
        }
        \caption{Otrzymane zdjęcie po nałożeniu maski: $\begin{bmatrix}
                                                            0    & 1/21 & 1/21 & 1/21 & 0 \\
                                                            1/21 & 1/21 & 1/21 & 1/21 & 1/21 \\
                                                            0    & 1/21 & 1/21 & 1/21 & 0
                                                        \end{bmatrix}$}
    \end{figure}



\section{Zadanie drugie}
\section{Zadanie trzecie}




\bibliographystyle{plainnat}
\bibliography{TexBase/Bibliography.bib}

\end{document}