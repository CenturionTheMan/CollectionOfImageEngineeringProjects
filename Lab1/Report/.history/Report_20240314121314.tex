\documentclass{article}
\input{TexBase/DocumentBase.tex}

\section{Wstęp}

\section{Zadanie pierwsze}
    \subsection{Cel ćwiczenia}
    Zadanie pierwsze polegało na zastosowaniu filtru górnoprzepustowego (tzw. detektora krawędzi) do obrazu. 
    W tym celu mieliśmy wykorzystać maskę:
    \[
        \begin{bmatrix}
            -1 & -1 & -1 \\
            -1 & 8 & -1 \\
            -1 & -1 & -1
        \end{bmatrix}
    \]

    \subsection{Teoria}
    Filtry obrazów są kluczowym narzędziem w dziedzinie przetwarzania obrazów, 
    służącym do modyfikacji wyglądu i charakterystyk obrazu poprzez różnorodne operacje matematyczne 
    na pikselach. Wśród podstawowych rodzajów filtrów wyróżnia się filtry górnoprzepustowe i dolnoprzepustowe.

    Filtry górnoprzepustowe, takie jak implementowany w zadaniu filtr Laplace'a, mają za zadanie podkreślać detale i krawędzie
    poprzez eliminację niskoczęstotliwościowych składowych obrazu i przepuszczanie wysokoczęstotliwościowych. 
    Działają na zasadzie mnożenia wartości pikseli przez odpowiednie współczynniki z maski, 
    co prowadzi do wyostrzenia obrazu i podkreślenia struktur.
    
    Z kolei filtry dolnoprzepustowe, np. filtr Gaussa, przepuszczają składowe niskoczęstotliwościowe, 
    eliminując wysokoczęstotliwościowe. Ich działanie polega na wygładzaniu obrazu poprzez średnią lub 
    ważoną wartość pikseli w otoczeniu. W efekcie uzyskuje się efekt rozmycia, który może być wykorzystywany 
    m.in. do redukcji szumów.

    Źródło: []\cite{Lubiński2007FiltrowanieObrazów}]


\section{Zadanie drugie}
\section{Zadanie trzecie}



\end{document}