\documentclass{article}
\input{TexBase/DocumentBase.tex}


\section{Zadanie pierwsze}
    \subsection{Cel ćwiczenia}
    Zadanie pierwsze polegało na zastosowaniu filtru górnoprzepustowego (tzw. detektora krawędzi) do obrazu. 
    W tym celu mieliśmy wykorzystać maskę:
    \[
        \begin{bmatrix}
            -1 & -1 & -1 \\
            -1 & 8 & -1 \\
            -1 & -1 & -1
        \end{bmatrix}
    \]

    \subsection{Teoria}
    Filtry obrazów są kluczowym narzędziem w dziedzinie przetwarzania obrazów, 
    służącym do modyfikacji wyglądu i charakterystyk obrazu poprzez różnorodne operacje matematyczne 
    na pikselach. Wśród podstawowych rodzajów filtrów wyróżnia się filtry górnoprzepustowe i dolnoprzepustowe.

    Filtry górnoprzepustowe, takie jak implementowany w zadaniu filtr Laplace'a, mają za zadanie podkreślać detale i krawędzie
    poprzez eliminację niskoczęstotliwościowych składowych obrazu i przepuszczanie wysokoczęstotliwościowych. 
    Działają na zasadzie mnożenia wartości pikseli przez odpowiednie współczynniki z maski, 
    co prowadzi do wyostrzenia obrazu i podkreślenia struktur. 
    
    Przykładowe filtry górnoprzepustowe to
    \begin{itemize}
        \item Filtr Laplace'a
        \item Filtr usuń średnią (ang. mean removal)
        \item Filtr HP1, HP2, HP3
    \end{itemize}

    
    Z kolei filtry dolnoprzepustowe, np. filtr Gaussa, przepuszczają składowe niskoczęstotliwościowe, 
    eliminując wysokoczęstotliwościowe. Ich działanie polega na wygładzaniu obrazu poprzez średnią lub 
    ważoną wartość pikseli w otoczeniu. W efekcie uzyskuje się efekt rozmycia, który może być wykorzystywany 
    m.in. do redukcji szumów.

    Przykładowe filtry dolnoprzepustowe to
    \begin{itemize}
        \item Filtr uśredniający
        \item Filtr kołowy
        \item Filtr piramidalny
    \end{itemize}
        
    Źródło: \cite{Lubinski2007}

    \subsection{Prezentacja wykonanego zadania}
    \begin{figure}[H]
        \centering
        \resizebox{\columnwidth}{!}{%
        \includegraphics{Img/zad1.png}%
        }
        \caption{Zdjęcie oryginalne i po nałożeniu filtru górnoprzepustowego podanego w zadaniu}
    \end{figure}

    \subsection{Wizualizację wybranych filtrów}


    \subsection{Eksperymenty z modyfikacją maski}

    \begin{enumerate}
        \item Obraz dla maski $\begin{bmatrix}
                                    1 & 1 & 1 \\
                                    1 & 8 & 1 \\
                                    1 & 1 & 1
                                \end{bmatrix}$
        \item Obraz dla maski $\begin{bmatrix}
                                    0 & 1/21 & 1/21 & 1/21 & 0 \\
                                    1/21 & 1/21 & 1/21 & 1/21 & 1/21 \\
                                    1/21 & 1/21 & 1/21 & 1/21 & 1/21 \\
                                    1/21 & 1/21 & 1/21 & 1/21 & 1/21 \\
                                    0 & 1/21 & 1/21 & 1/21 & 0
                                \end{bmatrix}$
    \end{enumerate}

    \begin{figure}[H]
        \centering
        \resizebox{\columnwidth}{!}{%
        \includegraphics{Img/zad1_1.png}%
        }
        \caption{Otrzymane zdjęcie po nałożeniu maski numer 1}
    \end{figure}

    \begin{figure}[H]
        \centering
        \resizebox{\columnwidth}{!}{%
        \includegraphics{Img/zad1_2.png}%
        }
        \caption{Otrzymane zdjęcie po nałożeniu maski numer 2}
    \end{figure}


\section{Zadanie drugie}
    \subsection{Cel ćwiczenia}
    Zadanie drugie polegało na przekształceniu koloru obrazu.
    Dla każdego piksela należało zastosować przekształcenie kanałów wzorem:

    \[
        \begin{bmatrix}
            R_{new} \\
            G_{new} \\
            B_{new}
        \end{bmatrix}
        =
        \begin{bmatrix}
            0.393 & 0.769 & 0.189 \\
            0.349 & 0.686 & 0.168 \\
            0.272 & 0.534 & 0.131
        \end{bmatrix}
        *
        \begin{bmatrix}
            R \\
            G \\
            B
        \end{bmatrix}
    \]

    Przedstawiona macierz nakłada efekt sepii na obraz.

    \subsection{Prezentacja wykonanego zadania}
    \begin{figure}[H]
        \centering
        \resizebox{\columnwidth}{!}{%
        \includegraphics{Img/zad2.png}%
        }
        \caption{Zdjęcie oryginalne i po nałożeniu efektu sepii}
    \end{figure}



\section{Zadanie trzecie}
    \section{Cel ćwiczenia}
    Zadanie trzecie polegało na skonwertowaniu obrazu z modelu RGB do modelu YCbCr i
    wyświetleniu poszczególnych kanałów Y, Cb, Cr w odcieniach szarości. Następnie należało
    przeprowadzić konwersję odwrotną i wyświetlić obraz wynikowy.

    Konwersję mieliśmy wykonać za pomocą wzoru:
    \[
        \begin{bmatrix}
            Y \\
            Cr \\
            Cb
        \end{bmatrix}
        =
        \begin{bmatrix}
            0 \\
            128 \\
            128
        \end{bmatrix}
        +
        \begin{bmatrix}
            0.229 & 0.587 & 0.114 \\
            0.500 & -0.418 & -0.082 \\
            -0.168 & -0.331 & 0.500
        \end{bmatrix}
        *
        \begin{bmatrix}
            R \\
            G \\
            B
        \end{bmatrix}
    \]

    \subsection{Teoria}
    Model YCrCb jest powszechnie używany w przetwarzaniu obrazów i wideo. 
    Składa się z trzech składowych: luminancji (Y), czerwoności (Cr) i niebieskości (Cb). 
    Kolor zielony jest wyznaczany matematycznie na podstawie powyższych składowych.
    MOdel ten, jest efektywny w przechowywaniu danych i pozwala na oddzielenie informacji o jasności 
    od informacji o kolorach. 

    \begin{itemize}
        \item Składowa Y (luminancja):
        \begin{itemize}
            \item Reprezentuje jasność piksela obrazu.
            \item Przechowuje informacje o intensywności światła lub luminancji dla każdego piksela.
        \end{itemize}
        \item Składowe Cr i Cb (chrominancja):
        \begin{itemize}
            \item Reprezentują informacje o kolorach pikseli obrazu.
            \item Składowa Cr odnosi się do różnicy między składową czerwieni a składową luminancji.
            \item Składowa Cb odnosi się do różnicy między składową niebieskości a składową luminancji.
            \item Te składowe są wykorzystywane do opisu kolorów w odniesieniu do osi luminancji, 
            co sprawia, że model jest efektywny i oszczędny w przechowywaniu informacji o kolorach.
        \end{itemize}
    \end{itemize}
    
    Źródła: [\cite{YCbCr1}], [/cite{YCbCr2}]


\bibliographystyle{plainnat}
\bibliography{TexBase/Bibliography.bib}

\end{document}