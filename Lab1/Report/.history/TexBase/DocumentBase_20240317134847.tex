%--PASTE INTO MAIN FILE--
% \documentclass{article}
% %--PASTE INTO MAIN FILE--
% \documentclass{article}
% %--PASTE INTO MAIN FILE--
% \documentclass{article}
% %--PASTE INTO MAIN FILE--
% \documentclass{article}
% \input{TexBase/DocumentBase.tex}
% \end{document}

\usepackage[margin = 0.7in]{geometry}
\usepackage{graphicx}
\usepackage{graphics}
\usepackage[T1]{fontenc}
\usepackage[polish]{babel}
\usepackage{cmap}
\usepackage[utf8]{inputenc}
\usepackage{float}
\usepackage{tabularx}
\usepackage[table,xcdraw]{xcolor}
\usepackage{lipsum}
\usepackage{titlesec}
\usepackage{minted}
\usepackage{xcolor}
\usepackage{caption}
\usepackage{enumitem}
\usepackage{csvsimple}
\usepackage{natbib}
\usepackage{blindtext}

\usepackage{amsmath} %math

\usepackage{numprint} % rounding
\usepackage[round-precision=3,round-mode=figures, scientific-notation=true]{siunitx} %scientific notation

\usepackage[hidelinks]{hyperref}
\usepackage{url}

\usepackage{bm} %bold for math


%TABS
\usepackage[]{booktabs}
\usepackage{tabularray}
\usepackage{multirow}

%\title{}
\author{Michał Dziedziak}
\date{\today}


\titlespacing\section{0pt}{12pt plus 4pt minus 2pt}{0pt plus 2pt minus 2pt}
\titlespacing\subsection{0pt}{12pt plus 4pt minus 2pt}{0pt plus 2pt minus 2pt}
\titlespacing\subsubsection{0pt}{12pt plus 4pt minus 2pt}{0pt plus 2pt minus 2pt}
\setlength{\parskip}{\baselineskip}%
\setlength{\parindent}{0pt}%

\newcommand{\squeezeup}{\vspace{-5mm}}


\begin{document}

\begin{titlepage}
    \begin{center}
        \vspace*{5cm}
        \rule{500pt}{1pt}\\
        \vspace*{0.5cm}
        \LARGE
        \textbf{Inżyniera Obrazów}\\
        \Large
        Laboratorium numer 2
        \vspace*{0.5cm}
        \rule{500pt}{1pt}
    \end{center}

    \vspace*{10cm}

    {\raggedright
        \large
        \textbf{Autor sprawozdania:} Michał Dziedziak 263901\\
        \textbf{Imię i Nazwisko prowadzącego kurs:} dr inż. Jan Nikodem\\
        \textbf{Dzień i godzina zajęć:} czwartek, 11:15 - 14:15
    }
\end{titlepage}


\tableofcontents
% \listoftables

%\renewcommand\listoflistingscaption{List of source codes}
% \listoflistings

\listoffigures


\newpage


% \begin{table}[H]
%     \centering
%     \begin{tabular}{|c|c|c|c|}%
%         \hline
%         \bfseries Numer iteracji & \bfseries Czas zalezienia rozwiązania [ms] & Koszt ścieżki & Błąd względny% specify table head
%         \csvreader[head to column names]{Csv/BestPathTest_SimulatedAnnealing_LINEAR_ftv47.csv}{}% use head of csv as column names
%         {\\\hline\Iteration & \num{\TimeInMiliSeconds} & \Cost & \num[round-precision=2, round-mode=places, scientific-notation=false]{\Error}\%}% specify your columns here
%         \\\hline    
%     \end{tabular}
%     \caption{}
%     \label{tab:}
% \end{table}

% \begin{figure}[H]
%     \centering
%     \resizebox{\columnwidth}{!}{%
%     \includegraphics{}%
%     }
%     \caption{}
%     \label{fig:}
% \end{figure}

% \begin{listing}[H]
%     \begin{minted}[frame=single,framesep=2mm,linenos,fontsize=\footnotesize]{language}
%         some code
%     \end{minted}
%     \caption{}
%     \label{lst:}
% \end{listing}


% \bibliographystyle{plainnat}
% \bibliography{TexBase/Bibliography}

% \end{document}

\usepackage[margin = 0.7in]{geometry}
\usepackage{graphicx}
\usepackage{graphics}
\usepackage[T1]{fontenc}
\usepackage[polish]{babel}
\usepackage{cmap}
\usepackage[utf8]{inputenc}
\usepackage{float}
\usepackage{tabularx}
\usepackage[table,xcdraw]{xcolor}
\usepackage{lipsum}
\usepackage{titlesec}
\usepackage{minted}
\usepackage{xcolor}
\usepackage{caption}
\usepackage{enumitem}
\usepackage{csvsimple}
\usepackage{natbib}
\usepackage{blindtext}

\usepackage{amsmath} %math

\usepackage{numprint} % rounding
\usepackage[round-precision=3,round-mode=figures, scientific-notation=true]{siunitx} %scientific notation

\usepackage[hidelinks]{hyperref}
\usepackage{url}

\usepackage{bm} %bold for math


%TABS
\usepackage[]{booktabs}
\usepackage{tabularray}
\usepackage{multirow}

%\title{}
\author{Michał Dziedziak}
\date{\today}


\titlespacing\section{0pt}{12pt plus 4pt minus 2pt}{0pt plus 2pt minus 2pt}
\titlespacing\subsection{0pt}{12pt plus 4pt minus 2pt}{0pt plus 2pt minus 2pt}
\titlespacing\subsubsection{0pt}{12pt plus 4pt minus 2pt}{0pt plus 2pt minus 2pt}
\setlength{\parskip}{\baselineskip}%
\setlength{\parindent}{0pt}%

\newcommand{\squeezeup}{\vspace{-5mm}}


\begin{document}

\begin{titlepage}
    \begin{center}
        \vspace*{5cm}
        \rule{500pt}{1pt}\\
        \vspace*{0.5cm}
        \LARGE
        \textbf{Inżyniera Obrazów}\\
        \Large
        Laboratorium numer 2
        \vspace*{0.5cm}
        \rule{500pt}{1pt}
    \end{center}

    \vspace*{10cm}

    {\raggedright
        \large
        \textbf{Autor sprawozdania:} Michał Dziedziak 263901\\
        \textbf{Imię i Nazwisko prowadzącego kurs:} dr inż. Jan Nikodem\\
        \textbf{Dzień i godzina zajęć:} czwartek, 11:15 - 14:15
    }
\end{titlepage}


\tableofcontents
% \listoftables

%\renewcommand\listoflistingscaption{List of source codes}
% \listoflistings

\listoffigures


\newpage


% \begin{table}[H]
%     \centering
%     \begin{tabular}{|c|c|c|c|}%
%         \hline
%         \bfseries Numer iteracji & \bfseries Czas zalezienia rozwiązania [ms] & Koszt ścieżki & Błąd względny% specify table head
%         \csvreader[head to column names]{Csv/BestPathTest_SimulatedAnnealing_LINEAR_ftv47.csv}{}% use head of csv as column names
%         {\\\hline\Iteration & \num{\TimeInMiliSeconds} & \Cost & \num[round-precision=2, round-mode=places, scientific-notation=false]{\Error}\%}% specify your columns here
%         \\\hline    
%     \end{tabular}
%     \caption{}
%     \label{tab:}
% \end{table}

% \begin{figure}[H]
%     \centering
%     \resizebox{\columnwidth}{!}{%
%     \includegraphics{}%
%     }
%     \caption{}
%     \label{fig:}
% \end{figure}

% \begin{listing}[H]
%     \begin{minted}[frame=single,framesep=2mm,linenos,fontsize=\footnotesize]{language}
%         some code
%     \end{minted}
%     \caption{}
%     \label{lst:}
% \end{listing}


% \bibliographystyle{plainnat}
% \bibliography{TexBase/Bibliography}

% \end{document}

\usepackage[margin = 0.7in]{geometry}
\usepackage{graphicx}
\usepackage{graphics}
\usepackage[T1]{fontenc}
\usepackage[polish]{babel}
\usepackage{cmap}
\usepackage[utf8]{inputenc}
\usepackage{float}
\usepackage{tabularx}
\usepackage[table,xcdraw]{xcolor}
\usepackage{lipsum}
\usepackage{titlesec}
\usepackage{minted}
\usepackage{xcolor}
\usepackage{caption}
\usepackage{enumitem}
\usepackage{csvsimple}
\usepackage{natbib}
\usepackage{blindtext}

\usepackage{amsmath} %math

\usepackage{numprint} % rounding
\usepackage[round-precision=3,round-mode=figures, scientific-notation=true]{siunitx} %scientific notation

\usepackage[hidelinks]{hyperref}
\usepackage{url}

\usepackage{bm} %bold for math


%TABS
\usepackage[]{booktabs}
\usepackage{tabularray}
\usepackage{multirow}

%\title{}
\author{Michał Dziedziak}
\date{\today}


\titlespacing\section{0pt}{12pt plus 4pt minus 2pt}{0pt plus 2pt minus 2pt}
\titlespacing\subsection{0pt}{12pt plus 4pt minus 2pt}{0pt plus 2pt minus 2pt}
\titlespacing\subsubsection{0pt}{12pt plus 4pt minus 2pt}{0pt plus 2pt minus 2pt}
\setlength{\parskip}{\baselineskip}%
\setlength{\parindent}{0pt}%

\newcommand{\squeezeup}{\vspace{-5mm}}


\begin{document}

\begin{titlepage}
    \begin{center}
        \vspace*{5cm}
        \rule{500pt}{1pt}\\
        \vspace*{0.5cm}
        \LARGE
        \textbf{Inżyniera Obrazów}\\
        \Large
        Laboratorium numer 2
        \vspace*{0.5cm}
        \rule{500pt}{1pt}
    \end{center}

    \vspace*{10cm}

    {\raggedright
        \large
        \textbf{Autor sprawozdania:} Michał Dziedziak 263901\\
        \textbf{Imię i Nazwisko prowadzącego kurs:} dr inż. Jan Nikodem\\
        \textbf{Dzień i godzina zajęć:} czwartek, 11:15 - 14:15
    }
\end{titlepage}


\tableofcontents
% \listoftables

%\renewcommand\listoflistingscaption{List of source codes}
% \listoflistings

\listoffigures


\newpage


% \begin{table}[H]
%     \centering
%     \begin{tabular}{|c|c|c|c|}%
%         \hline
%         \bfseries Numer iteracji & \bfseries Czas zalezienia rozwiązania [ms] & Koszt ścieżki & Błąd względny% specify table head
%         \csvreader[head to column names]{Csv/BestPathTest_SimulatedAnnealing_LINEAR_ftv47.csv}{}% use head of csv as column names
%         {\\\hline\Iteration & \num{\TimeInMiliSeconds} & \Cost & \num[round-precision=2, round-mode=places, scientific-notation=false]{\Error}\%}% specify your columns here
%         \\\hline    
%     \end{tabular}
%     \caption{}
%     \label{tab:}
% \end{table}

% \begin{figure}[H]
%     \centering
%     \resizebox{\columnwidth}{!}{%
%     \includegraphics{}%
%     }
%     \caption{}
%     \label{fig:}
% \end{figure}

% \begin{listing}[H]
%     \begin{minted}[frame=single,framesep=2mm,linenos,fontsize=\footnotesize]{language}
%         some code
%     \end{minted}
%     \caption{}
%     \label{lst:}
% \end{listing}


% \bibliographystyle{plainnat}
% \bibliography{TexBase/Bibliography}

% \end{document}

\usepackage[margin = 0.7in]{geometry}
\usepackage{graphicx}
\usepackage{graphics}
\usepackage[T1]{fontenc}
\usepackage[polish]{babel}
\usepackage{cmap}
\usepackage[utf8]{inputenc}
\usepackage{float}
\usepackage{tabularx}
\usepackage[table,xcdraw]{xcolor}
\usepackage{lipsum}
\usepackage{titlesec}
\usepackage{minted}
\usepackage{xcolor}
\usepackage{caption}
\usepackage{enumitem}
\usepackage{csvsimple}
\usepackage{natbib}
\usepackage{blindtext}

\usepackage{amsmath} %math

\usepackage{numprint} % rounding
\usepackage[round-precision=3,round-mode=figures, scientific-notation=true]{siunitx} %scientific notation

\usepackage[hidelinks]{hyperref}
\usepackage{url}

\usepackage{bm} %bold for math


%TABS
\usepackage[]{booktabs}
\usepackage{tabularray}
\usepackage{multirow}

%\title{}
\author{Michał Dziedziak}
\date{\today}


\titlespacing\section{0pt}{12pt plus 4pt minus 2pt}{0pt plus 2pt minus 2pt}
\titlespacing\subsection{0pt}{12pt plus 4pt minus 2pt}{0pt plus 2pt minus 2pt}
\titlespacing\subsubsection{0pt}{12pt plus 4pt minus 2pt}{0pt plus 2pt minus 2pt}
\setlength{\parskip}{\baselineskip}%
\setlength{\parindent}{0pt}%

\newcommand{\squeezeup}{\vspace{-5mm}}


\begin{document}

\begin{titlepage}
    \begin{center}
        \vspace*{5cm}
        \rule{500pt}{1pt}\\
        \vspace*{0.5cm}
        \LARGE
        \textbf{Inżyniera Obrazów}\\
        \Large
        Laboratorium numer 1
        \vspace*{0.5cm}
        \rule{500pt}{1pt}
    \end{center}

    \vspace*{10cm}

    {\raggedright
        \large
        \textbf{Autor sprawozdania:} Michał Dziedziak 263901\\
        \textbf{Imię i Nazwisko prowadzącego kurs:} dr inż. Jan Nikodem\\
        \textbf{Dzień i godzina zajęć:} 11:15 - 14:15
    }
\end{titlepage}


\tableofcontents
% \listoftables

%\renewcommand\listoflistingscaption{List of source codes}
% \listoflistings

\listoffigures


\newpage


% \begin{table}[H]
%     \centering
%     \begin{tabular}{|c|c|c|c|}%
%         \hline
%         \bfseries Numer iteracji & \bfseries Czas zalezienia rozwiązania [ms] & Koszt ścieżki & Błąd względny% specify table head
%         \csvreader[head to column names]{Csv/BestPathTest_SimulatedAnnealing_LINEAR_ftv47.csv}{}% use head of csv as column names
%         {\\\hline\Iteration & \num{\TimeInMiliSeconds} & \Cost & \num[round-precision=2, round-mode=places, scientific-notation=false]{\Error}\%}% specify your columns here
%         \\\hline    
%     \end{tabular}
%     \caption{}
%     \label{tab:}
% \end{table}

% \begin{figure}[H]
%     \centering
%     \resizebox{\columnwidth}{!}{%
%     \includegraphics{}%
%     }
%     \caption{}
%     \label{fig:}
% \end{figure}

% \begin{listing}[H]
%     \begin{minted}[frame=single,framesep=2mm,linenos,fontsize=\footnotesize]{language}
%         some code
%     \end{minted}
%     \caption{}
%     \label{lst:}
% \end{listing}


% \bibliographystyle{plainnat}
% \bibliography{Bibliography}
