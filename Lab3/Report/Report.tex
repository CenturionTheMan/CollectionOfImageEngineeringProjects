\documentclass{article}
\input{TexBase/DocumentBase.tex}



% \bibliographystyle{plainnat}
% \bibliography{TexBase/Bibliography}

\section{Temat laboratorium}
W ramach trzecich zajęć laboratoryjnych mieliśmy wykonać zadanie 4 i 5 z listy drugiej.
Punkty te dotyczyły implementacji algorytmu JPEG. 


\section{Zadania do wykonania i plan pracy}

    \subsection{Zadanie 4: Implementacja części algorytmu JPEG}
    Zadanie czwarte zakładało uproszczoną implementację algorytmu JPEG. W ramach tego punktu mieliśmy
    zaimplementować:
    \begin{itemize}
        \item Kroki: 0, 1, 2, 3, 7, 8 algorytmu JPEG. Gdzie:
        \begin{itemize}
            \item Krok 0: Wczytanie obrazu wejściowego,
            \item Krok 1: Konwersja modelu barw: RGB -> YCbCr,
            \item Krok 2: Przeskalowanie w dół macierzy składowych Cb i Cr,
            \item Krok 3: Podział na bloki o rozmiarze 8x8,
            \item Krok 7: Zwinięcie każdego bloku 8x8 do wiersza 1x64 - algorytm ZigZag,
            \item Krok 8: Zakodowanie danych obrazu,
        \end{itemize}
        \item Zmierzyć liczbę bajtów powstałego obrazu po kroku 8
        \item Ocenić wpływ kroku 2. na rozmiar i wygląd, poprzez stworzenie trzech wariancji obrazu:
        \begin{itemize}
            \item bez próbkowania,
            \item z próbkowaniem ci drugi element,
            \item z próbkowaniem co czwarty element.
        \end{itemize}
        \item Dokonać dekompresji poprzez odwrócenie powyższych kroków.
    \end{itemize}

    \subsection{Zadanie 5: Dokończenie implementacji algorytmu JPEG}
    Zadanie piąte zakładało dokończenie implementacji algorytmu JPEG. W ramach tego punktu mieliśmy
    zaimplementować:
    \begin{itemize}
        \item Pozostałe kroki algorytmu:
        \begin{itemize}
            \item Krok 4: Wykonanie dyskretnej transformacji cosinusowej na każdym bloku obrazu,
            \item Krok 5: Podzielenie każdego bloku obrazu przez macierz kwantyzacji,
            \item Krok 6: Zaokrąglenie wartości w każdym bloku do liczb całkowitych.
        \end{itemize}
        \item Ocenić jak wybór czynnika QF wpływa na rozmiar i wygląd obrazka.
        \item Dokonać dekompresji poprzez odwrócenie powyższych kroków.
    \end{itemize}
    
\section{Teoria}

\end{document}