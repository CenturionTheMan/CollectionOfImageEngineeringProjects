\documentclass{article}
\input{TexBase/DocumentBase.tex}

\section{Temat laboratorium}
Steganografia jest dziedziną nauki zajmującą się ukrywaniem komunikatów w jawnym medium.
W ramach tego laboratorium, eksplorujemy techniki steganograficzne,
koncentrując się na metodzie ukrywania informacji w obrazach poprzez modyfikację ich najmniej znaczących bitów (LSB - Least Significant Bit).

W praktyce używamy metody, która zapisuję ukrytą informację w obrazie poprzez modyfikację n-wybranych
ostatnich bitów pikseli obrazu.

\section{Zadanie 1}
\subsection{Treść}
Pierwsze zadanie polegało o użyciu gotowych funkcji \textit{hide\_message} oraz \textit{reveal\_message} w
celu zakodowania i odkodowania wiadomości w obrazie.

Zadanie zostało rozbudowane o menu, umożliwiające wybór czynności: \textit{zakodowanie wiadomości},
\textit{odkodowanie wiadomości}.
Pozwala ono na zapisania obrazu z zakodowaną wiadomością do pliku, oraz odczytanie wiadomości z obrazu z
pliku, który może pochodzić z innego programu.

\subsection{Prezentacja wykonanego zadania}

\subsubsection*{Kodowanie wiadomości}
\begin{figure}[H]
    \centering
    \resizebox{\columnwidth}{!}{%
        \includegraphics{img/zad1_2.png}%
    }
    \caption{Zdjęcie konsoli podczas kodowania wiadomości}
\end{figure}

\begin{figure}[H]
    \centering
    \resizebox{\columnwidth}{!}{%
        \includegraphics{img/zad1.png}%
    }
    \caption{Zdjęcie z zakodowaną wiadomością}
\end{figure}

\subsubsection*{Dekodowanie wiadomości}
\begin{figure}[H]
    \centering
    \resizebox{\columnwidth}{!}{%
        \includegraphics{img/zad1_3.png}%
    }
    \caption{Zdjęcie konsoli podczas dekodowania wiadomości}
\end{figure}



\section{Zadanie 2}
\subsection{Treść}
W zadaniu drugim nalezało zakodować w obrazie wiadomość o długości 75\% liczby bajtów w obrazie.
Następnie należało zmierzyć wartości MSE pomiędzy oryginalnym obrazem o obrazem z zakodowaną wiadomością
dla wartości \textit{nbits} od 1 do 8.

\subsection{Prezentacja wykonanego zadania}
\begin{figure}[H]
    \centering
    \resizebox{\columnwidth}{!}{%
        \includegraphics{img/zad2.png}%
    }
    \caption{Obrazy z kolejnymi wartościami nbits}
\end{figure}

\begin{figure}[H]
    \centering
    \resizebox{\columnwidth}{!}{%
        \includegraphics{img/zad2_plot.png}%
    }
    \caption{Wykres zależności MSE od wartości nbits}
\end{figure}

% TODO wnioski?


\section{Zadanie 3}
\subsection{Treść}
W zadaniu trzecim należało zmodyfikować funkcję \textit{hide\_message} i \textit{reveal\_message} tak,
aby można było wybrać od którego miejsca ma być zapisywana wiadomość w obrazie.

Zadanie zostało rozbudowane o menu analogiczne do tego z zadania pierwszego.

% TODO jakoś opisać jak to zrobiłem
\subsection{Prezentacja wykonanego zadania}

\subsubsection*{Kodowanie wiadomości}
\begin{figure}[H]
    \centering
    \resizebox{\columnwidth}{!}{%
        \includegraphics{img/zad3_2.png}%
    }
    \caption{Zdjęcie logów konsoli podczas procesu kodowania wiadomości}
\end{figure}

\begin{figure}[H]
    \centering
    \resizebox{\columnwidth}{!}{%
        \includegraphics{img/zad3.png}%
    }
    \caption{Obraz przed i po zakodowaniu wiadomości od pozycji 50000}
\end{figure}


\subsubsection*{Dekodowanie wiadomości}
\begin{figure}[H]
    \centering
    \resizebox{\columnwidth}{!}{%
        \includegraphics{img/zad3_3.png}%
    }
    \caption{Zdjęcie logów konsoli podczas procesu dekodowania wiadomości}
\end{figure}


\section{Zadanie 4}
\subsection{Treść}
Zadanie czwarte polegało na zaimplementowaniu funkcjonalności ukrywania (i odzyskiwania) obrazu w innym
obrazie.

W praktyce oznaczało to dodanie kroku, który spłaszczał piksele w obrazie do jednego wymiaru, a następnie
ukrywał kolejne bity w najmniej znaczących bitach obrazu docelowego.

Analogicznie proces odzyskiwania obrazu polegał na odczytaniu ukrytych bitów i odpowiednią ich transformację
do kanałów RGB.

Zadanie zostało rozbudowane o menu analogiczne do tego z zadania pierwszego.

\subsection{Prezentacja wykonanego zadania}

\subsubsection*{Kodowanie obrazu}
\begin{figure}[H]
    \centering
    \resizebox{\columnwidth}{!}{%
        \includegraphics{img/zad4_2.png}%
    }
    \caption{Zdjęcie logów konsoli podczas procesu ukrywania obrazu w innym obrazie}
\end{figure}

\begin{figure}[H]
    \centering
    \resizebox{\columnwidth}{!}{%
        \includegraphics{img/zad4.png}%
    }
    \caption{Prezentacja obrazu przed i po zakodowaniu w nim innego obrazu}
\end{figure}

\subsubsection*{Dekodowanie obrazu}
\begin{figure}[H]
    \centering
    \resizebox{\columnwidth}{!}{%
        \includegraphics{img/zad4_4.png}%
    }
    \caption{Zdjęcie logów konsoli podczas procesu odzyskiwania obrazu z innego obrazu}
\end{figure}

\begin{figure}[H]
    \centering
    \resizebox{\columnwidth}{!}{%
        \includegraphics{img/zad4_3.png}%
    }
    \caption{Prezentacja obrazu z ukrytym obrazem oraz odkodowanego obrazu}
\end{figure}

\section{Zadanie 5}
\subsection{Treść}
W zadaniu piątym należało rozbudować odczytywanie obrazu z pliku o automatyczne wykrywanie końca wiadomości.
W ramach zadania zaimplementowałem rozpoznawanie końca wiadomości poprzez sprawdzanie czy aktualnie
przetwarzane bajty pokrywają się ze stopką konca pliku jpg:
\[\text{kod stopki} =1111 1111 1101 1001_2 = FFD9_{16}\]

\subsection{Prezentacja wykonanego zadania}
\begin{figure}[H]
    \centering
    \resizebox{\columnwidth}{!}{%
        \includegraphics{img/zad5_2.png}%
    }
    \caption{Zdjęcie logów konsoli podczas procesu odzyskiwania obrazu z innego obrazu}
\end{figure}

\begin{figure}[H]
    \centering
    \resizebox{\columnwidth}{!}{%
        \includegraphics{img/zad5.png}%
    }
    \caption{Prezentacja obrazu z ukrytym obrazem oraz odkodowanego obrazu}
\end{figure}

\end{document}