\documentclass{article}
\input{TexBase/DocumentBase.tex}

\section{Temat laboratorium}
Steganografia jest dziedziną nauki zajmującą się ukrywaniem komunikatów w jawnym medium.
W ramach tego laboratorium, eksplorujemy techniki steganograficzne,
koncentrując się na metodzie ukrywania informacji w obrazach poprzez modyfikację ich najmniej znaczących bitów (LSB - Least Significant Bit).

W praktyce używamy metody, która zapisuję ukrytą informację w obrazie poprzez modyfikację n-wybranych
ostatnich bitów pikseli obrazu.

\section{Zadanie 1}
\subsection{Treść}
Pierwsze zadanie polegało o użyciu gotowych funkcji \textit{hide\_message} oraz \textit{reveal\_message} w
celu zakodowania i odkodowania wiadomości w obrazie.

Zadanie zostało rozbudowane o menu, umożliwiające wybór czynności: \textit{zakodowanie wiadomości},
\textit{odkodowanie wiadomości}.
Pozwala ono na zapisania obrazu z zakodowaną wiadomością do pliku, oraz odczytanie wiadomości z obrazu z
pliku, który może pochodzić z innego programu.

\subsection{Prezentacja wykonanego zadania}
\begin{figure}[H]
    \centering
    \resizebox{\columnwidth}{!}{%
        \includegraphics{img/zad1_2.png}%
    }
    \caption{Logi z konsoli podczas kodowania wiadomości w obrazie}
\end{figure}

\begin{figure}[H]
    \centering
    \resizebox{\columnwidth}{!}{%
        \includegraphics{img/zad1.png}%
    }
    \caption{Prezentacja obrazu z zakodowaną wiadomością}
\end{figure}

\begin{figure}[H]
    \centering
    \resizebox{\columnwidth}{!}{%
        \includegraphics{img/zad1_3.png}%
    }
    \caption{Logi z konsoli podczas dekodowania wiadomości w obrazie}
\end{figure}


\section{Zadanie 2}
\subsection{Treść}
W zadaniu drugim nalezało zakodować w obrazie wiadomość o długości 75\% liczby bajtów w obrazie.
Następnie należało zmierzyć wartości MSE pomiędzy oryginalnym obrazem o obrazem z zakodowaną wiadomością
dla wartości \textit{nbits} od 1 do 8.

\subsection{Prezentacja wykonanego zadania}
\begin{figure}[H]
    \centering
    \resizebox{\columnwidth}{!}{%
        \includegraphics{img/zad2.png}%
    }
    \caption{Obrazy z kolejnymi wartościami nbits}
\end{figure}

\begin{figure}[H]
    \centering
    \resizebox{\columnwidth}{!}{%
        \includegraphics{img/zad2_plot.png}%
    }
    \caption{Wykres zależności MSE od wartości nbits}
\end{figure}

% TODO wnioski?


\section{Zadanie 3}
\subsection{Treść}
W zadaniu trzecim należało zmodyfikować funkcję \textit{hide\_message} i \textit{reveal\_message} tak,
aby można było wybrać od którego miejsca ma być zapisywana wiadomość w obrazie.

% TODO jakoś opisać jak to zrobiłem

\subsection{Prezentacja wykonanego zadania}


\section{Zadanie 4}
\subsection{Treść}
\subsection{Prezentacja wykonanego zadania}


\section{Zadanie 5}
\subsection{Treść}
\subsection{Prezentacja wykonanego zadania}


\end{document}